\chapter{Customização Unioeste}

\section{Lista de códigos}

Usado para criar a lista de códigos, adicionar sintaxe highlight, enumerar as linhas e colorir o fundo, para dar destaque a implementação.

Sintaxe básica:
\begin{verbatim}
\begin{codigo}[!htb]
    \caption{Espaço para o título do código}
    \label{Espaço para o label do código, para ser usado na referência}  
    \begin{lstlisting}[language = Linguagem de programação a ser usada]
        <CÓDIGO>
    \end{lstlisting}
\end{codigo}
\end{verbatim}

\begin{codigo}[h]
  \caption{Programa hello.c}
  \label{codigoC}
    \lstinputlisting[language=c]{Codigos/hello.c}
\end{codigo}

\begin{codigo}[h]
  \caption{Programa em Java}
  \label{codigoJava}
    \lstinputlisting[language=java]{Codigos/exjava.java}
\end{codigo}

\begin{codigo}[h]
  \caption{Programa em Python}
  \label{codigoPython}
    \lstinputlisting[language=python]{Codigos/teste.py}
\end{codigo}
