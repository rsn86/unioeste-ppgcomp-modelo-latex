% resumo em português
\setlength{\absparsep}{18pt} % ajusta o espaçamento dos parágrafos do resumo
\begin{resumo}

\comocitar{OLIVEIRA JÚNIOR}{Javan Ataíde de} %não remover! Parametros: SOBRENOME, Primeiro Nome e Nomes do Meio.
 % Ex. OLIVEIRA JÚNIOR, Javan Ataíde de. Compressão de dados em redes LoRa: Um compromisso entre desempenho e consumo de energia. Orientador: Dr. Marcio Seiji Oyamada. 2021. 191 f. Dissertação (Mestrado em Ciência da Computação) - Universidade Estadual do Oeste do Paraná, Cascavel – Paraná, 2021.
 
Segundo a \citeonline[3.1-3.2]{NBR6028:2003}, o resumo deve ressaltar o objetivo, o método, os resultados e as conclusões do documento. A ordem e a extensão destes itens dependem do tipo de resumo (informativo ou indicativo) e do tratamento que cada item recebe no documento original. O resumo deve ser precedido da referência do documento, com exceção do resumo inserido no próprio documento. (\ldots) As palavras-chave devem figurar logo abaixo do resumo, antecedidas da expressão Palavras-chave:, separadas entre si por ponto e finalizadas também por ponto.

 \textbf{Palavras-chave}: latex. abntex. editoração de texto.
\end{resumo}

% abstract
\begin{resumo}[Abstract]
 \begin{otherlanguage*}{english}
	\howtocite{OLIVEIRA JÚNIOR}{Javan Ataide de} %não remover!
 	%parametros: SOBRENOME, Primeiro Nome e Nomes do Meio.
 	% Ex. OLIVEIRA JÚNIOR, Javan Ataíde de. Título em Lingua estrangeira. Orientador: Dr. Marcio  	Seiji Oyamada. 2021. 191 f. Dissertação (Mestrado em Ciência da Computação) - Universidade Estadual do Oeste do Paraná, Cascavel – Paraná, 2021.
 	
   This is the english abstract.

   \vspace{\onelineskip}
 
   \noindent 
   \textbf{Keywords}: latex; abntex; text editoration
 \end{otherlanguage*}
\end{resumo}